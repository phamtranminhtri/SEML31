\section{Kết luận}

\subsection{Tóm tắt công việc}
Báo cáo này trình bày một hệ thống hoàn chỉnh cho bài toán nhận diện hợp âm từ tín hiệu âm thanh sử dụng Hidden Markov Model và thuật toán Viterbi. Các đóng góp chính bao gồm:

\begin{enumerate}
    \item \textbf{Pipeline xử lý tín hiệu số phong phú}: Từ tín hiệu âm thanh thô đến vector đặc trưng 75 chiều kết hợp nhiều loại features (chroma CQT, tonnetz, spectral contrast) cùng delta features để bắt động thái thay đổi. Áp dụng HPSS để tách harmonic, median filtering để giảm nhiễu, và L2 normalization. Tham số DSP: sample rate 22050 Hz, hop length 512.
    
    \item \textbf{Mô hình hoá HMM supervised với 25 trạng thái}: Ước lượng tham số HMM ($\pi$, $A$) từ dữ liệu có nhãn với Laplace smoothing ($\epsilon = 10^{-8}$). Sử dụng GMM adaptive (1-3 components tùy số samples) với full covariance để mô hình hoá emission probabilities.
    
    \item \textbf{Giải mã Viterbi hiệu quả}: Implementation thuật toán Viterbi trong miền log-probability để tránh underflow, kết hợp median filter (kernel size 5) post-processing để làm mịn chuỗi dự đoán.
    
    \item \textbf{Thiết kế modular functions}: Các hàm \texttt{extract\_features()}, \texttt{load\_labels()}, \texttt{calc\_hmm\_parameters()}, \texttt{train\_GMM()}, \texttt{viterbi\_log()} được tổ chức thành modules độc lập, dễ tái sử dụng và test.
    
    \item \textbf{Đánh giá định lượng}: Thiết lập train/test split (40/10 bài, seed=72), tính frame-level accuracy, và visualize confusion matrix để phân tích chi tiết hiệu năng trên 25 trạng thái.
\end{enumerate}

\subsection{Ý nghĩa và ứng dụng}
Hệ thống này có thể được ứng dụng trong:
\begin{itemize}
    \item \textbf{Phân tích âm nhạc tự động}: Hỗ trợ nhạc sĩ, nhà nghiên cứu âm nhạc học phân tích cấu trúc hợp âm của bản nhạc.
    \item \textbf{Hệ thống học nhạc}: Tự động tạo chord chart cho người học guitar, piano.
    \item \textbf{Tạo nhạc tự động (music generation)}: Cung cấp thông tin hợp âm làm điều kiện cho mô hình sinh nhạc.
    \item \textbf{Music Information Retrieval (MIR)}: Làm nền tảng cho các tác vụ như tìm kiếm bài hát theo progression, phân loại thể loại dựa trên harmony.
\end{itemize}

\subsection{Bài học kinh nghiệm}
\begin{itemize}
    \item \textbf{Tầm quan trọng của tham số DSP}: Lựa chọn sample rate và hop length ảnh hưởng trực tiếp đến chất lượng đặc trưng và hiệu suất tính toán. Cần cân bằng giữa độ phân giải thời gian (43 frames/giây) và chi phí tính toán.
    
    \item \textbf{Multi-feature fusion hiệu quả}: Kết hợp chroma (pitch content), tonnetz (harmonic relationships), spectral contrast (timbre) cùng delta features (temporal dynamics) cho accuracy tốt hơn đáng kể so với chỉ dùng chroma. Vector 75 chiều capture được nhiều khía cạnh của âm nhạc.
    
    \item \textbf{Trade-off giữa độ phức tạp mô hình và dữ liệu}: Rút gọn không gian trạng thái về 25 (chỉ maj/min) giúp tránh overfitting với dataset nhỏ (50 bài). Mở rộng thêm chord types cần dataset lớn hơn.
    
    \item \textbf{Smoothing quan trọng}: Laplace smoothing với $\epsilon = 10^{-8}$ (rất nhỏ) vào ma trận $A$ và $\pi$ giúp tránh $\log(0)$ trong Viterbi mà không làm bias model quá nhiều. Giá trị quá lớn sẽ làm mất đi thông tin từ data.
    
    \item \textbf{GMM adaptive hiệu quả}: Điều chỉnh số components (1-3) theo số samples của mỗi trạng thái tránh overfitting cho rare chords. Full covariance matrix (75×75) bắt được correlation giữa các features nhưng tốn memory.
    
    \item \textbf{Post-processing quan trọng}: Median filter sau Viterbi loại bỏ outliers (chord nhảy đột ngột) hiệu quả hơn mean filter vì giữ được biên. Kernel size 5 là trade-off tốt giữa smoothing và preserving transitions.
\end{itemize}

\subsection{Hướng phát triển tương lai}
Các hướng nghiên cứu tiếp theo có thể bao gồm:
\begin{enumerate}
    \item \textbf{Mở rộng không gian trạng thái}: Thêm các chord types phổ biến (7th, sus4, dim, aug) với hierarchical vocabulary hoặc factorized representations (root × quality). Cần augment dataset hoặc transfer learning để tránh overfitting.
    
    \item \textbf{Kết hợp deep learning end-to-end}: Sử dụng Transformer hoặc Bi-LSTM trực tiếp trên spectrogram/CQT, với CTC loss hoặc attention mechanism. Hoặc hybrid approach: CNN extract features → HMM/CRF temporal modeling.
    
    \item \textbf{Transfer learning cross-dataset}: Pre-train trên dataset lớn đa dạng (McGill Billboard, Isophonics, JAAH) để học robust representations, sau đó fine-tune hoặc few-shot learning cho domain cụ thể (genre-specific, artist-specific).
    
    \item \textbf{Duration modeling explicit}: Dùng Hidden Semi-Markov Model (HSMM) hoặc add duration features để mô hình hoá chord duration distribution, tránh chuyển đổi quá nhanh (minimum duration constraints).
    
    \item \textbf{Beat-synchronous analysis}: Aggregate features theo beats (từ beat tracker) thay vì fixed hop length. Chord changes thường align với beats, giúp reduce noise và tăng musical consistency.
    
    \item \textbf{Segment-level metrics}: Thêm đánh giá segment-level (oversegmentation, undersegmentation, boundary precision/recall) theo MIREX standards, không chỉ frame-level accuracy. Weighted metrics theo chord frequency.
    
    \item \textbf{Real-time streaming}: Optimize với Cython/numba, use incremental Viterbi (online decoding), và look-ahead buffer nhỏ. Target latency < 100ms cho live performance applications.
    
    \item \textbf{Uncertainty quantification}: Output confidence scores cho mỗi prediction (từ Viterbi probabilities hoặc GMM posteriors), giúp downstream applications handle uncertain regions.
\end{enumerate}

\subsection{Lời kết}
Báo cáo này đã trình bày chi tiết một hệ thống nhận diện hợp âm hoàn chỉnh sử dụng Hidden Markov Model, từ lý thuyết nền tảng (DSP, HMM, Viterbi) đến implementation cụ thể với Python và thư viện mã nguồn mở (librosa, scikit-learn, scipy). Các điểm nổi bật:

\begin{itemize}
    \item \textbf{Feature engineering comprehensive}: Kết hợp 3 loại features (chroma, tonnetz, spectral contrast) với delta/delta-delta, tạo vector 75 chiều rich hơn nhiều so với chỉ dùng chroma 12 chiều.
    
    \item \textbf{Modeling careful}: HMM 25 states với GMM adaptive (1-3 components), Laplace smoothing chính xác ($10^{-8}$), và median filter post-processing cho kết quả stable.
    
    \item \textbf{Evaluation rigorous}: Train/test split có reproducibility (seed=72), frame-level accuracy, confusion matrix analysis trên Beatles dataset (50 songs).
\end{itemize}

Mặc dù phương pháp HMM/GMM là "classical" so với deep learning hiện đại, nó vẫn là nền tảng quan trọng giúp hiểu sâu về:
\begin{itemize}
    \item Bản chất sequential của âm nhạc (temporal dependencies)
    \item Trade-offs trong model design (state space size, feature dimensionality, model complexity vs. data size)
    \item Probabilistic reasoning và dynamic programming
\end{itemize}

Code implementation modular và well-documented, dễ dàng extend cho các tác vụ tương tự (key detection, structure analysis). Hy vọng tài liệu này hữu ích cho người đọc quan tâm đến Music Information Retrieval, signal processing, và probabilistic graphical models.
