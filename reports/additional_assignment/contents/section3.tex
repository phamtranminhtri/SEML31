\section{Kết quả và đánh giá}

\subsection{Cấu hình thực nghiệm}

\subsubsection{Dataset}
\begin{itemize}
    \item Dữ liệu huấn luyện: Tập các file âm thanh (.mp3/.wav) kèm file annotation (.lab)
    \item Mỗi file .lab chứa thông tin: \texttt{start\_time end\_time chord\_label}
    \item Sau khi aggregate, dữ liệu được lưu trong file \texttt{hmm\_data\_package.npz} gồm:
    \begin{itemize}
        \item \texttt{X\_data}: Ma trận chroma $(M_{total} \times 12)$, với $M_{total}$ là tổng số frame từ tất cả các bài
        \item \texttt{y\_data}: Vector nhãn $(M_{total},)$ chứa state ID
        \item \texttt{lengths}: Mảng chứa số frame của từng bài, để phân tách các chuỗi độc lập
    \end{itemize}
\end{itemize}

\subsubsection{Tham số mô hình}
\begin{itemize}
    \item Số trạng thái HMM: $N = 26$ (24 hợp âm + N + Complex)
    \item Sample rate: 22050 Hz
    \item Hop length: 512 samples ($\approx$ 23.2 ms, $\approx$ 43 frames/giây)
    \item Số thành phần GMM mỗi trạng thái: 40 (có thể điều chỉnh dựa trên BIC/AIC analysis)
    \item Covariance type: \texttt{full}
\end{itemize}

\subsection{Quy trình huấn luyện}

\subsubsection{Bước 1: Trích xuất và căn chỉnh}
\begin{verbatim}
all_chroma_vectors, all_label_vectors = 
    aggregate_all_data(DATA_DIR, audio_processor)
\end{verbatim}
Kết quả: Danh sách các cặp (chroma features, aligned labels) cho mỗi bài hát.

\subsubsection{Bước 2: Tính toán tham số HMM}
\begin{itemize}
    \item \textbf{Ma trận $A$}: Đếm số lần chuyển trạng thái, thêm smoothing $10^{-5}$, chuẩn hoá
    \item \textbf{Vector $\pi$}: Đếm trạng thái đầu tiên của mỗi bài, chuẩn hoá
    \item \textbf{GMM cho mỗi trạng thái}: Tập hợp tất cả chroma vector thuộc trạng thái đó, huấn luyện GMM
\end{itemize}

Output:
\begin{verbatim}
INITIAL_STATE_DISTRIBUTION shape: (26,)
TRANSITION_MATRIX shape: (26, 26)
EMISSION_MODELS: 26 GMM models
\end{verbatim}

\subsection{Quy trình dự đoán}

Với một file âm thanh mới (không có nhãn):

\subsubsection{Bước 1: Trích xuất chroma}
\begin{verbatim}
features = audio_processor.extract_chroma(audio_file)
\end{verbatim}
Output: Ma trận chroma $(M \times 12)$ với $M$ là số frame.

\subsubsection{Bước 2: Áp dụng Viterbi}
\begin{verbatim}
predicted_state_ids = viterbi(features, 
                               INITIAL_STATE_DISTRIBUTION, 
                               TRANSITION_MATRIX, 
                               EMISSION_MODELS)
\end{verbatim}
Output: Mảng state ID $(M,)$.

\subsubsection{Bước 3: Chuyển đổi về nhãn hợp âm}
\begin{verbatim}
predicted_chords = [ID_TO_CHORD[state_id] 
                    for state_id in predicted_state_ids]
\end{verbatim}
Output: Danh sách nhãn hợp âm (ví dụ: \texttt{['C:maj', 'C:maj', 'F:maj', ...]}).

\subsubsection{Bước 4: Lưu kết quả}
Mỗi frame tương ứng một nhãn, được ghi ra file \texttt{.txt}:
\begin{verbatim}
C:maj
C:maj
F:maj
G:maj
...
\end{verbatim}

Nếu muốn chuyển về dạng khoảng thời gian (segment), cần gom các frame liên tiếp có cùng nhãn và tính thời gian bằng công thức:
\begin{equation}
t_i = \frac{i \times \text{hop\_length}}{\text{sample\_rate}}
\end{equation}

\subsection{Phân tích kết quả}

\subsubsection{Thống kê hợp âm dự đoán}
Sau khi merge tất cả file dự đoán, notebook thực hiện phân tích đơn giản:
\begin{itemize}
    \item Đếm tần suất xuất hiện của mỗi hợp âm
    \item Hiển thị top 10 hợp âm phổ biến nhất
    \item Tổng số frame được phân loại
\end{itemize}

\textbf{Nhận xét chung:}
\begin{itemize}
    \item Các hợp âm phổ biến (C:maj, G:maj, F:maj, A:min, D:min) thường xuất hiện nhiều hơn
    \item Trạng thái "N" (no-chord) có thể chiếm tỷ lệ đáng kể nếu có nhiều đoạn im lặng hoặc intro/outro
    \item Trạng thái "Complex" xuất hiện ít hơn, phản ánh việc đơn giản hoá không gian trạng thái
\end{itemize}

\subsubsection{Ưu điểm của phương pháp}
\begin{itemize}
    \item \textbf{Tận dụng thông tin chuỗi}: HMM mô hình hoá tiến trình hợp âm tự nhiên (ví dụ: I-IV-V-I), giúp dự đoán mượt mà hơn so với phân loại độc lập từng frame.
    \item \textbf{Xử lý nhiễu tốt}: GMM với nhiều thành phần có thể bắt được sự đa dạng của cùng một hợp âm trên nhiều nhạc cụ, phong cách khác nhau.
    \item \textbf{Hiệu quả tính toán}: Viterbi với $O(T \times N^2)$ khả thi cho real-time hoặc near real-time với $N=26$ và $T$ vài nghìn frame.
    \item \textbf{Supervised learning đơn giản}: Không cần thuật toán phức tạp như Baum-Welch, chỉ cần đếm tần suất từ dữ liệu có nhãn.
\end{itemize}

\subsubsection{Hạn chế và hướng cải thiện}
\begin{itemize}
    \item \textbf{Không gian trạng thái rút gọn}: 26 trạng thái không thể biểu diễn các hợp âm phức tạp (7th, 9th, sus4, dim, aug). Có thể mở rộng thêm trạng thái nhưng cần dữ liệu lớn hơn.
    \item \textbf{Giả định Markov bậc 1}: Không bắt được cấu trúc dài hạn (ví dụ: verse-chorus pattern). Có thể sử dụng higher-order HMM hoặc kết hợp với mô hình cấu trúc âm nhạc.
    \item \textbf{Phụ thuộc vào chất lượng annotation}: Nếu file .lab có lỗi hoặc không đồng nhất, tham số HMM sẽ bị ảnh hưởng.
    \item \textbf{Tuning GMM components}: Số thành phần 40 là một lựa chọn ban đầu. Có thể tối ưu hoá bằng cross-validation hoặc grid search với BIC/AIC (code có phần comment về điều này).
    \item \textbf{Đánh giá định lượng}: Hiện tại chưa có metric như accuracy, precision, recall trên test set. Nên tách train/test và tính các chỉ số đánh giá.
\end{itemize}

\subsection{Đề xuất cải tiến}
\begin{enumerate}
    \item \textbf{Mở rộng tập trạng thái}: Thêm các hợp âm 7th, suspended, diminished phổ biến (ví dụ: C:maj7, G:7, D:sus4).
    \item \textbf{Duration modeling}: HMM chuẩn không mô hình hoá độ dài trạng thái (duration). Có thể dùng Hidden Semi-Markov Model (HSMM) để mô hình hoá rõ ràng thời gian tồn tại của mỗi hợp âm.
    \item \textbf{Context features}: Thêm các đặc trưng ngoài chroma (ví dụ: MFCC, spectral contrast) để cải thiện khả năng phân biệt.
    \item \textbf{Deep learning hybrid}: Kết hợp với mạng neural (ví dụ: CNN hoặc RNN) để học biểu diễn tốt hơn, sau đó dùng HMM để làm mịn chuỗi.
    \item \textbf{Ensemble methods}: Kết hợp nhiều mô hình (HMM với tham số khác nhau, hoặc HMM + classification) để tăng độ tin cậy.
\end{enumerate}
